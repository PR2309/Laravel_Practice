% Use the Ctrl+Alt+B shortcut to compile the .tex file into a PDF.
% Use the Ctrl+Alt+V shortcut to preview the PDF in the built-in viewer.

% Basics of LaTeX
% 
% LaTeX is a typesetting system commonly used for technical and scientific documents. It is based on the TeX typesetting system, which was created by Donald Knuth. LaTeX provides a high-level language for creating professional documents, including academic papers, theses, and resumes.
% 
% Basics: https://www.overleaf.com/learn/latex/Learn_LaTeX_in_30_minutes
% LaTeX documentation: https://www.latex-project.org/help/documentation/
% 
% 
% Step 1: Installation
% The LaTeX Workshop extension requires a LaTeX distribution to compile .tex files. You need to install one of the following:
% (i) MiKTeX (Windows): https://miktex.org/download
% (ii) TeX Live (Linux/Mac): https://www.tug.org/texlive/
% (iii) MacTeX (Mac): https://www.tug.org/mactex/
% 
% 
% Step 2: After installing the LaTeX distribution, verify that it’s correctly installed and accessible from the command line:
%   Open a terminal or command prompt.
%   Run the following command:
%       latexmk --version
% If the command returns a version number, the installation is successful.
% 
% 
% Step 3: Adding Path:
%   Add the path to the LaTeX distribution's bin directory (e.g., C:\Program Files\MiKTeX 2.9\miktex\bin\x64 for MiKTeX).
%   Add the following line(Linux):
%       export PATH=/usr/local/texlive/2023/bin/x86_64-linux:$PATH
%   (Replace the path with the correct one for your TeX Live installation.)
%   Save the file and run source ~/.bashrc (or the appropriate file) to apply the changes.
% 
% 
% Step 4: LaTeX Workshop Extension:
%   The LaTeX Workshop extension provides a live preview of the compiled document. It also allows you to compile the .tex file into a PDF using the Ctrl+Alt+B shortcut.
% 
% Open a terminal and edit your shell configuration file (e.g., .bashrc, .zshrc, or .bash_profile).
% 
% 
% Step 5: Test LaTeX Workshop
% Open your .tex file in VS Code.
% Use the Ctrl+Alt+B shortcut to compile the file.
% Use the Ctrl+Alt+V shortcut to preview the PDF.
% 




% cv.tex
\documentclass[11pt]{article}
\usepackage[margin=1in]{geometry}
\usepackage{enumitem}

% Title
\title{\textbf{John Doe's CV}}
\author{}
\date{}

\begin{document}

\maketitle

% Contact Information
\section*{Contact Information}
\begin{itemize}[leftmargin=*]
    \item \textbf{Email:} john.doe@example.com
    \item \textbf{Phone:} +1 (123) 456-7890
    \item \textbf{LinkedIn:} linkedin.com/in/johndoe
    \item \textbf{GitHub:} github.com/johndoe
\end{itemize}

% Education
\section*{Education}
\subsection*{University of Example}
\textbf{Bachelor of Science in Computer Science} \hfill \textit{Graduated: May 2023} \\
GPA: 3.8/4.0 \\
Relevant Coursework: Data Structures, Algorithms, Machine Learning

% Experience
\section*{Experience}
\subsection*{Software Engineer Intern}
\textbf{Example Tech Company} \hfill \textit{Summer 2022} \\
\begin{itemize}[leftmargin=*]
    \item Developed a web application using React and Node.js.
    \item Optimized database queries, reducing response time by 30\%.
    \item Collaborated with a team of 5 engineers using Agile methodologies.
\end{itemize}

% Skills
\section*{Skills}
\begin{itemize}[leftmargin=*]
    \item \textbf{Programming Languages:} Python, JavaScript, Java, C++
    \item \textbf{Tools:} Git, Docker, VS Code, LaTeX
    \item \textbf{Languages:} English (Fluent), Spanish (Intermediate)
\end{itemize}


\end{document}